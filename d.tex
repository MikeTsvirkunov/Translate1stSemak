\documentclass[a4paper,10pt]{article}

\usepackage{cmap}					
\usepackage[T2A]{fontenc}			
\usepackage[utf8]{inputenc}			
\usepackage[english,russian]{babel}
\usepackage{caption}
\usepackage[left=2.5cm, top=2cm, right=1cm, bottom=20mm, nohead, nofoot]{geometry}

\begin{document}
    \section*{Обработка естественного языка для определения случая
    Факторы судебного разбирательства по защите детей}

    \subsection*{Ввкдение}
    \par
    Файлы социальных дел содержат подробные сведения о жизни и потребностях уязвимых групп. 
    Однако традиционные исследования по анализу случаев для получения обобщаемых знаний требуют значительных ресурсов, и поэтому размеры выборки ограничены. 
    Появление технологии «больших данных» и обширных централизованных электронных хранилищ записей предлагает исследователям в области социальной работы новые альтернативы, 
    включая нахождение связи в данных и прогнозирование рисков с использованием проверреной информации.
    Но документы с произвольным текстом, включая оценки, отчеты и хронологию дел, остаются в значительной степени неиспользованным ресурсом. 
    В этой статье описывается, как 5000 заявлений суда по социальным вопросам, хранящихся в Службе консультативной поддержки судов по делам детей и семьи в Англии, были проанализированы с использованием обработки естественного языка (NLP) на основе простых правил и математических принципов. 
    Тринадцать факторов, связанных с вредом и риском для детей, участвующих в процессах ухода за детьми в Англии, были определены с помощью автоматизированных компьютерных технологий, и почти 90\% согласия с профессионалами были достигнуты, когда факторы были четкими. 
    Исследование представляет собой инновационный подход к исследованию социальной работы по сложным социальным проблемам. 
    В добавок в статье обсуждаются моменты обучения; практические последствия; направления будущих исследований; и технические и этические проблемы \textit{NLP}. 
    Появление «больших данных» продемонстрировало потенциал и недостатки методов, используемых для понимания и управления различными областями человеческой деятельности, включая предпочтения клиентов, оптимизацию трафика, политические дебаты и здравоохранение. «Большие данные» до сих пор не имеют четкого определения, но часто понимаются с точки зрения «трех V»: 
        объем -- растущие хранилища информации; 
        разнообразие -- широкий спектр типов данных для поиска связи; 
        скорость -- скорость накапливания данных для предоставления информации в реальном времени с возможностью мгновенного обновления. 
    В социальных науках «большие данные» можно рассматривать с точки зрения «расширения доступности новых типов данных, которые ранее не были доступны для исследований в области социальных наук». 
    Важно отметить, что данные» «находятся» (т.е. в существуют в базах данных) а не «создают» (с помощью традиционных методов исследования), что дает социальной работе главное преимущество -- работа в реальном времени, не тратя больших затрат на исследования, связанных с их сбором для каждого исследования. 
    В сфере здравоохранения и социальной помощи понимание риска во все большей степени формируется на основе вероятностной информации и все чаще выводится из больших хранилищ данных. 
    Они могут быть связаны и анонимными сторонними агентствами. 
    Например, путем объединения наборов данных о процедурах ухода, с медицинской информацией и показателями депривации, возникают взаимосвязи со здоровьем родителей. 
    Моделирование прогнозных рисков, применяемое к связанным наборам данных, также может использоваться на практике, например, при принятии решений о проверке защиты детей.
    \par\hrulefill\par
    Защита детей от жестокого обращения является ключевым аспектом социальной помощи, который поднимает этические вопросы, такие как права человека. 
    Подходы к большим данным могут смягчить их, способствуя согласованности, справедливости и, следовательно, прозрачности для пользователей. 
    «Большие данные» имеют свои недостатки. 
    Существуют особые проблемы с извлечением значимой информации из больших данных в некоторых контекстах, включая здравоохранение и социальную помощь. 
    Для социологов системы, как правило, не оптимизированы для облегчения извлечения и анализа, что делает данные более «беспорядочными», чем «систематическими» в том, как они были собраны для исследовательских целей. 
    В рамках социальной работы большая часть деталей дела записывается в виде описательного текста, а не числовых данных, что создает серьезную проблему для существующих технологий. 
    Хотя существуют инициативы по количественной оценке того, что ранее было чисто качественными текстовыми данными, представляется вероятным, что анализ неструктурированных текстов потребуется для наиболее полной реализации потенциала больших данных в ближайшем будущем, т. данные обработки. 
    Один из способов добиться этого -- обработка естественного языка (NLP). \textit{NLP} относится к компьютерным технологиям, которые извлекают структурированные данные из неструктурированного языка, письменного или устного. 
    Чем более автоматизированным это может стать, тем эффективнее и полезнее будет процесс. 
    \textit{NLP}, несмотря на бурное развитие в бизнесе и медицине, остается относительно неразвитым в социальной работе и социальной помощи. 
    Поскольку \textit{NLP}, как правило, использует передовые методы информатики, оно может отпугнуть многих исследователей от такой технической области. 
    Однако эффективное \textit{NLP} требует не только владения компьютерными технологиями, но и глубокого понимания текстового материала и его предметно-специфического контекста. 
    В этой статье описывается применение \textit{NLP} в рамках социальной работы по защите детей, которое устраняет разногласия между социальными и компьютерными науками за счет применения профессиональных знаний наряду с простыми статистическими методами \textit{NLP}, которые могут быть более доступными для ученых, не занимающихся информатикой. 
    В документе рассматривается компонент \textit{NLP} более крупного исследования, в котором изучался ряд факторов, влияющих на результаты для детей, в отношении которых социальные работники обращаются с заявлениями об изъятии из-под родительской опеки. 
    Факторы на уровне случая (Случайности), то есть риск и вред для детей, были получены с использованием \textit{NLP}, тогда как переменные процесса и демографические переменные были извлечены из административных наборов данных и опубликованных государственных источников. 
    Поскольку \textit{NLP} представляет собой инновацию для исследований в области социальной работы, мы рассмотрим две области, прежде чем осветить методологию и результаты исследования. 
    Во-первых, мы исследуем потенциальные преимущества, которые \textit{NLP} может предложить для практики и исследований в области социальной работы. 
    Во-вторых, мы описываем техники и принципы \textit{NLP} для целевой аудитории, не являющейся компьютерными специалистами.
    \par\hrulefill\par
    На сегодняшний день применение \textit{NLP} в социальной работе ограничено, однако многие исследования из смежных областей демонстрируют потенциал в этом контексте. 
    Здесь мы обсуждаем их под четырьмя широкими заголовками: практическое руководство; описание услуг; предсказательная точность; прозрачность и справедливость. 
    Чтобы разобраться в обширной документации, профессионалы должны разработать стратегии или личные «эмпирические правила». 
    Какие ключевые документы и их составные части заслуживают наибольшего внимания? 
    Этот неизбежный подход «скорочтения» может привести к тому, что важная и актуальная информация будет упущена из виду. 
    Различные решения, которые принимают профессионалы, также будут основываться на уже сформированных взглядах на случай или полученных устных сообщениях и могут отражать предвзятость подтверждения. 
    Мунро отметил неоправданное доверие практикующих врачей в оценках «доказательствам, которые были яркими, конкретными, вызывающими эмоции и либо первой, либо последней полученной информацией». 
    Письменная информация воспринималась реже, чем устная. 
    В одном случае все специалисты, присутствовавшие на осмотре, проглядели неслучайную травму, зарегистрированную в отчете педиатра, который все читали. 
    Хотя эти комментарии отражают практику, наблюдавшуюся более двух десятилетий назад, они, вероятно, актуальны и сегодня, поскольку, как сообщается, давление на социальных работников возрастает из-за увеличения нагрузки и сложности дел. 
    Технология обобщения текста с использованием \textit{NLP} направлена на выявление ключевых моментов, указание «красных флажков», руководство и подсказку к действиям. 
    Они представляют собой системы сдержек, противовесов и гарантий, которые могут предотвратить неблагоприятные последствия. 
    Описание услуги Изучение потребностей пользователей услуг, Bako et al. успешно применили \textit{NLP} к электронным медицинским картам. 
    Они измерили количество пациентов, посещающих медицинские центры и получающих поддержку социальной работы, а также характер и количество проведенных вмешательств. 
    Авторы пришли к выводу, что алгоритмы NLP могут быть включены в существующие системы с небольшими затратами для более эффективного нацеливания на ресурсы с использованием данных, которые в противном случае были бы скрыты в свободном тексте. 
    Чжоу и др. (2015) применили \textit{NLP} к выписке пациентов, чтобы определить 20\% случаев депрессии, которые не были закодированы вручную. 
    В других исследованиях также использовались медицинские записи для количественной оценки областей социального риска, например изоляции и бездомности, которые влияют на результаты лечения в популяции пациентов. 
    Эти исследования показывают, как \textit{NLP} потенциально может улучшить понимание потребностей и трудностей тех, кто получает поддержку социальной работы.
    \par\hrulefill\par
    Предоставляя большой объем «помеченных» отчетов, компьютер может определить для себя ключевые слова, кластеры и ассоциации, связанные с «правильным» ответом, то есть «связано или не связано со злоупотреблением психоактивными веществами». 
    Чтобы понять сильные стороны и ограничения каждого подхода, необходимо рассмотреть контекст и цели. 
    Методы, основанные на правилах, могут привнести предвзятость со стороны человека, в то время как машины беспристрастно обнаруживают ассоциации слов и точно настраивают вес, придаваемый каждому из них. 
    Выполняя тысячи итераций, компьютеры «обучаются» и выдают все более точные результаты. 
    Однако создание такого типа искусственного интеллекта требует огромного количества помеченных данных, которых в некоторых контекстах может не быть. 
    Например, машинное обучение может быть оптимальным подходом для определения тем, связанных с пятизвездочными отзывами клиентов, когда доступны тысячи, а мнения клиентов неизвестны. 
    Машинное обучение также может быть уместно в контексте социальной работы для оценки факторов, связанных с известным результатом, например, принятие или отклонение направления. 
    Если, однако, цель состояла в том, чтобы оценить распространенность определенного фактора из невидимых текстов документов, многим оценщикам, возможно, пришлось бы сначала прочитать и оценить документы, и они все равно могли бы не согласиться. 
    Также нельзя полностью избежать человеческого суждения и потенциальной предвзятости с помощью более технических решений. 
    В то время как машины могут возвращать кластеры слов, связанные с заданными результатами, люди должны интерпретировать и определять темы, которые представляют сгенерированные компьютером кластеры. 
    Неконтролируемые подходы могут быть наиболее ценными, когда темы являются новыми, например, когда реакция клиентов на новый продукт неизвестна. 
    Однако в такой области, как социальная работа, типы трудностей и диапазон потребностей, с которыми обычно сталкиваются работники сферы услуг, хорошо известны специалистам-практикам и лицам, определяющим политику, и иногда также предписываются в рамках оценок. 
    В этом случае использование неконтролируемого подхода может принести мало сюрпризов, в то время как подход, основанный на правилах, может быть прагматичным и экономически эффективным. 
    Исследователь сначала определяет факторы, указанные в литературе и профессиональных данных, а затем использует простые правила, которые можно последовательно применять ко всем обрабатываемым документам, чтобы идентифицировать конкретные факторы, присутствующие в текстах документов. 
    Правила, установленные людьми, в отличие от технологии машинного обучения, также прозрачны, поскольку можно объяснить, как они были созданы и реализованы. Это является решающим соображением, когда оценки и прогнозы используются для принятия важных решений о жизни людей (Рудин, 2019).
    \end{document}


